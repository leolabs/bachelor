%%%%%%%%%%%%%%%%%%%%%%%%%%%%%%%%%%%%%%%%%%%%%%%%%%%%%%%%%%%%%%%
%
% Beispiel für eine Bachelorarbeit mit FH Aachen FB 8
% Titelblattstyles
%
%	Prof. Enning, 18.07.2013
%   Überarbeitet, 17.06.2014
%
%%%%%%%%%%%%%%%%%%%%%%%%%%%%%%%%%%%%%%%%%%%%%%%%%%%%%%%%%%%%%%%

%%%%%%%%%%%%%% Präambel %%%%%%%%%%%%%%%%%%%%%%%%%%%%%%%%%%%%%%%%
\documentclass [
% 	draft		% falls ohne Bilder gedruckt werden soll (Entwurf)
	]{scrbook}	% KOMA Klasse für Bücher
%
\usepackage{fhacmb}	% Style-File für Titelblatt. Ggf. bei Enning holen
%
% Einstellen der Optionen für die KOMA Klasse
\KOMAoptions{
	parskip=true,		% Absätze mit Abstand
	fontsize=12pt,		% Standardschriftgröße
	toc=flat,		% Inhaltsverzeichnis ohne Einzüge
	twoside=false,		% Einseitig setzen
	numbers=nodotatend,	% Nummerierungen nicht mit Punkt abschließen
% die folgenden Optionen nehmen die entsprechende Dinge ins Inhaltsverzeichnis auf
% mit der bei texlive vorhandenen aktuellen Version von pdflatex funkioniert es nicht mehr
% (bekannter Bug)
	toc=bibliography,	% Literaturverzeichnis ins Inhaltsverzeichnis
	toc=listof,		% Abbildung- und Tabellenverzeichnis ins Inhaltsverzeichnis
%	toc=index,		% Stichwortverzeichnis ins Inhaltsverzeichnis
	}
%
%%%%%% Immer benötigte Packages
%
\providecommand{\tightlist}{\setlength{\itemsep}{0pt}\setlength{\parskip}{0pt}}
\makeatletter
\def\maxwidth{\ifdim\Gin@nat@width>\linewidth\linewidth\else\Gin@nat@width\fi}
\makeatother
% Scale images if necessary, so that they will not overflow the page
% margins by default, and it is still possible to overwrite the defaults
% using explicit options in \includegraphics[width, height, ...]{}
\setkeys{Gin}{width=\maxwidth,height=8cm,keepaspectratio}
\usepackage[style=authoryear-comp,backend=biber,maxcitenames=1]{biblatex}
\addbibresource{../bibliography/bibliography.bib}
\usepackage[T1]{fontenc}		% sonst funktioniert die Silbentrennung bei Umlauten nicht
\usepackage[utf8]{inputenc}	% Eingabedekodierung. Ermöglicht Umlaute. Achtung: Unbedingt mit Betreuer
				% Verwendung der Umlaute-Eingabemethode absprechen. Im Zweifel \"O für Ö
\usepackage{csquotes}
\usepackage[ngerman]{babel}	% Silbentrennung und Sprachanpassung
\usepackage{blindtext}		% Blindtext
\usepackage[hidelinks]{hyperref}		% Sprungmarken, z.B. im Inhaltsverzeichnis auf Textpassagen
\usepackage{textcomp}		% Sonst funktioniert z.B. \texteuro nicht
\usepackage[automark]{scrlayer-scrpage}	% Package zum Definieren der Kopf- und Fußzeilen
\usepackage{amsmath}		% Muss sein
\usepackage{mathrsfs} 	% Weitere Mathematik-Symbole, z.B. Laplace-L
\usepackage{grffile}

\usepackage{graphicx}		% Definiert o.a. \includegraphics
\graphicspath{{../content.textbundle/}{.}}

%%%%% Anpassung an Formatvorlagen des Fachbereichs

\usepackage{helvet}		% Serifenlose Schrift ähnlich Arial
\renewcommand{\familydefault}{\sfdefault}	% als Standardschrift serifenlose Schrift verwenden

\usepackage{geometry} 		% Ränder direkt einstellen
\geometry{a4paper, top=30mm, left=40mm, right=20mm, bottom=25mm} % nach Vorgabe
\linespread{1.5} 		% Zeilenabstand nach Vorgabe

\usepackage[font=small,labelfont=bf]{caption}  % customization of captions of floating environments such figure and table
\DeclareCaptionLabelFormat{blank}{}

\usepackage{chngcntr}		% Ändert Verhalten von Countern
\counterwithout{figure}{section}	% Figure-Nummerierung nicht bei section-Wechsel zurücksetzen
\renewcommand{\thefigure}{\thechapter.\arabic{figure}}	% im Stil 3.2

\renewcommand{\thetable}{\thechapter.\arabic{table}}
\renewcommand{\theequation}{\thechapter.\arabic{equation}}

%%%% für das Erzeugen von Grafiken mit Zeichenbefehlen
\usepackage{tikz}		% Grundpaket
\usetikzlibrary{shapes,arrows}	% einige Symbolpackages
\usepackage{tikz-cd}		% einige Symbolpackages

%%%%%% Gegebenenfalls nützliche Zusatzpackages

%\usepackage{blindtext}		% Blindtexte zum Ausprobieren von Formatierungen
%\usepackage{color}		% Schriftfarben
\usepackage{colortbl}		% für die Hintergrundfarbe von Tabellen
%\usepackage{gensymb}		% Definiert Formelzeichen, die im math und im text-Modus gleich aussehen
%\usepackage{wrapfig}		% Definiert die wrapfigure-Umgebung (Bild am Rand von Text umflossen)
%\usepackage{pdfpages}		% Einbinden von pdf-Seiten
%
%\usepackage{booktabs} 		% Schöne Tabellen
%\usepackage{tabu}	 	% Sehr einfach Tabellen gestalten
%\usepackage{array} 		% Erweiterung der Tabellenumgebung, neue Spaltentypen
\usepackage{paralist}		% Weitere Nummeriungsoptionen, z.b. alphabetisch für enumerate/itemize
%\usepackage{verbatim}		% Verbesserte verbatim-Umgebung (z.b. Programm-Listings)
%\usepackage{subfig}		% Unterfigures mit eigenen Bildunterschriften
%

%%%%%% Sammelsurium
%
%\renewcommand{\labelitemii}{$\circ$} % Bullets für itemize-Listen
%
%%%%%%%%%%  Angaben für Titelseite %%%%%%%%%%%%%%%%%%%%%%
%
% Angaben für Titelseite
\arbeitstyp{Bachelorarbeit}
\fachbereich{Elektrotechnik und Informationstechnik}
\studiengang{Media and Communications for Digital Business}
\titel{Component Sprints: Entwicklung eines neuartigen Ansatzes zur \linebreak\mbox{Optimierung} des Übergangs vom Prototypen zum MVP}
\autor{Leo Bernard}
\matrnr{3069756}
\betreuer{Marco Motullo}
\datum{17. Juni 2014}
\dank{Besonderer Dank gilt Daniel Wirtz, René Nauheimer, Marcus Weiner und Moritz Gunz}

%
%%%%%%%%%%%%%%%%%%%%%%%%%%%%%%%%%%%%%%%%%%%%%%%%%%%%%%%%%%%
%

\newcounter{savepage}
\begin{document}
% Einige Anpassungen müssen nach \begin{document} stehen !!
\renewcaptionname{ngerman}{\figurename}{Abb.}
\renewcaptionname{ngerman}{\tablename}{Tabelle}
\renewcaptionname{ngerman}{\contentsname}{Inhalt}% Inhalt statt Inhaltsverzeichnis

%\crefname{figure}{Abb.}{Abb.}
%\crefname{equation}{Gleichung}{Gleichung}
%\crefname{table}{Tabelle}{Tabelle}

%%%%%%%%%%%%%%%%%%%%%%%%%%%%%%%%%%%%%%%%%%%%%%%%%%%%%%%%%%%%
% Titel im FH Style
%%%%%%%%%%%%%%%%%%%%%%%%%%%%%%%%%%%%%%%%%%%%%%%%%%%%%%%%%%%%
\fhacmbtitle{\includegraphics[height=4cm]{fh_logo.png}}{5pt}{5pt}

%%%%%%%%%%%%%%%%%%%%%%%%%%%%%%%%%%%%%%%%%%%%%%%%%%%%%%%%%%%%
% Erklärung / Geheimhaltung
%%%%%%%%%%%%%%%%%%%%%%%%%%%%%%%%%%%%%%%%%%%%%%%%%%%%%%%%%%%%

%\frontmatter 	% Wenn der Hauptteil mit Seite 1 beginnen soll
\pagestyle{plain}
% Ein Kapitel das nicht mitgezählt wird beginnen
\chapter*{Eidesstattliche Erklärung}

% Diese Seite soll keine Kopf-/Fußzeile enthalten
\thispagestyle{empty}

Hiermit erkläre ich, dass ich die vorliegende Arbeit selbstständig angefertigt habe. Es wurden nur die in der Arbeit ausdrücklich benannten Quellen und Hilfsmittel benutzt. Wörtlich oder sinngemäß übernommenes Gedankengut habe ich als solches kenntlich gemacht.

% Etwas Abstand setzen
\vspace{1.5cm}

% Bereich für Datum, Name und Unterschrift
\begin{center}
\begin{tabular}[h]{lp{2cm}p{5.5cm}}
 \\
\cline{1-1}\cline{3-3}
Ort, Datum& & Leo Bernard\\
\end{tabular}
\end{center}

%%%%%%%%%%%%%%%%%%%%%%%%%%%%%%%%%%%%%%%%%%%%%%%%%%%%%%%%%%%%
% Inhaltsverzeichnis
%%%%%%%%%%%%%%%%%%%%%%%%%%%%%%%%%%%%%%%%%%%%%%%%%%%%%%%%%%%%

\pagenumbering{gobble}
\tableofcontents

\listoffigures
\pagenumbering{Roman}
\setcounter{page}{1}
\listoftables
\setcounter{savepage}{\arabic{page}}

\mainmatter	% Wenn der Hauptteil mit Seite 1 beginnen soll
\pagestyle{plain}
%%%%%%%%%%%%%%% Anpassung des Seitenstils an FH-Layoutvorschrift %%%%%%%%%%%%
\renewcommand{\chaptermark}[1]{\markboth{\thechapter\hspace{1cm}#1}{}}	% Kapitel für Headerzeile neu definieren (ohne Nummer)
\chead{}		% Header Mitte löschen
\ihead{\leftmark}	% Kapitelbezeichnung links setzen
\renewcommand{\headfont}{\bfseries}	% Bold-Font für Headerzeile verwenden
\setheadsepline{0.5pt}

% \input{aufgabenstellung}
\input{../content.textbundle/text.tex}

\printbibliography

\appendix
% \input{anhang}

\end{document}